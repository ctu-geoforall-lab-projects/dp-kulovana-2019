\chapter{Použité technologie}
\label{3-technologie}

Třetí kapitola stručně představuje jednotlivé technologie použité při
tvorbě webového administrátorského rozhraní.

\section{LDAP}
% https://en.wikipedia.org/wiki/Lightweight_Directory_Access_Protocol

Lightweight Directory Access Protocol neboli \zk{LDAP} je otevřený, standardizovaný protokol. Slouží k přístupu k datům, jejich úpravám a ukládání na adresářovém serveru přes Internet Protocol (\zk{IP}). Je vhodný pro správu informací o uživatelích. Je nezávislý na operačním systému.

Protokol LDAP vychází ze standardu X.500, jehož je odlehčenou variantou. Někdy je nazýván X.500-lite.\footnote{https://www.webopedia.com/TERM/L/LDAP.html}

\subsection{OpenLDAP}
\label{openldap}

% neznam licenci
\begin{figure}[H] \centering
      \includegraphics[width=100pt]{./pictures/LDAPlogo.png}
      \caption[OpenLDAP logo]{OpenLDAP logo (zdroj:
\href{http://www.openldap.org/images/headers/LDAPlogo.gif}{OpenLDAP.org})}
      \label{fig:ldap}
  \end{figure}

OpenLDAP je otevřená (open-source) implementace protokolu \zk{LDAP} vyvíjená pod (hlavičkou) OpenLDAP Project. OpenLDAP je distribuován pod vlastní licencí OpenLDAP Public License.\footnote{http://www.openldap.org/software/release/license.html}

\subsection{django-python3-ldap}
https://github.com/etianen/django-python3-ldap/blob/master/README.rst

\subsection{ldap3}
https://ldap3.readthedocs.io/welcome.html


\newpage

\section{Python}

\begin{figure}[H] \centering
      \includegraphics[width=150pt]{./pictures/python-logo-master-v3-TM.png}
      \caption[Python logo]{Python logo (zdroj:
\href{https://www.python.org/static/community_logos/python-logo-master-v3-TM.png}{Python.org})}
      \label{fig:python}
  \end{figure}
  
% (3) https://docs.python.org/3/faq/general.html#general-python-faq

Python je vysokoúrovňový, interpretovaný programovací jazyk. Podporuje
procedurálně i objektově orientované programování, je výkonný, zároveň
má velmi jednoduchou a čistou syntax. V ostatních jazycích je
odsazování řádků doporučeno z hlediska přehlednosti, u Pythonu je
základním stavebním kamenem a je povinné.\cite{Kulovana, 2017}

Dnes je Python vyvíjen jako open source projekt
pod záštitou neziskové organizace Python Software Foundation
(\zk{PSF}). Je distribuován pod licencí \zk{PSF}, která je
kompatibilní s \zk{GPL}. Je možné ho nainstalovat na běžné platformy
jako Windows, Unix nebo Mac OS, pro Linux je většinou součástí
základní instalace. Při vyvarování se systémově závislých funkcí je
přenositelný mezi~platformami bez jakýchkoli změn.

Python má široké využití, od jednoduchých programů po rozsáhlé
aplikace. Právě pro tyto možnosti, univerzálnost, přehlednost kódu a
výkonnost z něj udělaly programovací jazyk, který je mezi začátečníky ve
velké oblibě. Během krátké doby v~něm funkční skript zvládne napsat
každý.

--

%https://wiki.python.org/moin/Python2orPython3#Which_version_should_I_use.3F
% http://python-notes.curiousefficiency.org/en/latest/python3/questions_and_answers.html#when-can-we-expect-python-2-to-be-a-purely-historical-relic

Python v současnosti existuje ve dvou hlavních verzích - Python 2 a Python 3. Python 3.0 byl vydán v roce 2008\footnote{https://www.python.org/downloads/} a není zpětně kompatibilní s verzí Python 2. Hlavním důvodem pro takto zásadní změnu bylo rozhodnutí Guido van Rossuma, zakladatele jazyku Python, očistit Python 2.x od mnoha problémů pořádně v jednom kroku. 

Největšími změnami jsou upravení tradičních tříd a oddělení abstrakcí \texttt{řetězec} a \texttt{posloupnost bytů}. Textový řetězec (\texttt{str}) je nově převeden ve výchozím nastavení na typ unicode, což vytváří konzistentnější a spolehlivější prostředí. Změny doznalo chování operátoru dělení \texttt{/} celých čísel. Výsledkem je číslo s plovoucí desetinnou čárkou, na rozdíl od předchozí verze, která vracela celé číslo (ve verzi 3.x dostupné pod operátorem \texttt{//}). Mezi nejviditelnější novinky pro běžného uživatele je přechod od příkazu \texttt{print} k funkci \texttt{print()}. 

Python 3 je obecně přívětivější k učení nových uživatelů a je považován za budoucnost tohoto jazyka. Stále však existuje mnoho programů, jež fungují na poslední verzi Python 2.7 vydané v roce 2010\footnote{https://www.python.org/downloads/}, která již nedostává žádné velké aktualizace. Důvody jsou různé - daný projekt byl vyvíjen v Pythonu 2 a nejsou dostatečné kapacity na jeho přechod na novější verzi; na některých operačních systémech není Python 3 nainstalovaný a uživatelé nemají vždy práva si ho sami doinstalovat; existuje potřeba využívat externí knihovnu, která podporuje pouze Python 2 a není triviální ji převést do Pythonu 3. 

Poslední vydanou stabilní verzí je Python 3.7.\footnote{https://www.python.org, květen 2019} 

\section{Django}

\begin{figure}[H] \centering
      \includegraphics[width=150pt]{./pictures/django-logo-positive.png}
      \caption[Django logo]{Django logo (zdroj:
\href{https://static.djangoproject.com/img/logos/django-logo-positive.png}{Djangoproject.com})}
      \label{fig:django}
  \end{figure}

% http://www.moreware.org/books/The%20Definitive%20Guide%20to%20Django.pdf

Django je vysokoúrovňový webový framework napsaný v jazyce Python. Je udržované organizací Django Software Foundation (\zk{DSF}), bezplatné a vydané pod open-source licencí \zk{BSD}. Název získalo po jazzovém kytaristovi Djangovi Reinhardtovi.

Hlavním cílem Djanga je usnadnit tvorbu komplexních, databází řízených webových aplikací. Pro tento účel se řídí zásadou oddělení zodpovědností (angl. Separation of concerns) a volně navazuje na architekturu Model-view-controller (\zk{MVC}). Ta sestává ze tří volně propojených komponent:
\begin{itemize}
\item model (model) - reprezentace dat, k nimž aplikace přistupuje
\item view (pohled) - uživatelské rozhraní
\item controller (řadič) - reakce na žádosti a zajištění změn v pohledu nebo v modelu
\end{itemize}

Poslední vydanou stabilní verzí v době zpracování bylo Django 2.1 a veškerý další popis uvedený níže platí pro tuto verzi.

Frameworky mají snahu co největší množství práce automatizovat a tak při vytvoření projektu pomocí příkazu:

\begin{center}
\texttt{django-admin startproject nazev\_projektu}
\end{center}

vznikne automaticky jeho základní kostra, která slouží jako podstata již fungující Django aplikace.

\dirtree{%
.1 project.			
	.2 mysite.
		.3 \_\_init\_\_.py.
		.3 settings.py.
		.3 urls.py.
		.3 wsgi.py.
	.2 db.sqlite3.
	.2 manage.py.
}

\subsubsection{\_\_init\_\_.py}
Soubor \textit{\_\_init\_\_.py} dává Pythonu najevo, že s adresářem, v němž se soubor nachází, má být zacházeno jako s balíčkem modulů Pythonu. Jedná se o prázdný soubor, který se obvykle nijak nemění.

\subsubsection{settings.py}
\label{settings}
Soubor \textit{settings.py} skrývá nastavení Django projektu, např. jakým způsobem má probíhat autentizace, kde jsou umístěné další potřebné soubory či informace o použitých databázích a registrovaných aplikacích.

\subsubsection{urls.py}
\textit{urls.py} obsahuje \zk{URL} cesty pro vytvořený Django projekt, v podstatě se jedná o jakýsi rejstřík stránky. Implicitně obsahuje cestu k vestavěné administrátorské konzoli. Při propojení s Django aplikacemi jsou sem přidány další \zk{URL} adresy.

\subsubsection{wsgi.py}
\zk{WSGI} je specifikace popisující komunikaci mezi webovým serverem a webovou aplikací nebo frameworkem v jazyce Python. Jedná se o primární nástroj nasazení programů v Djangu. Soubor \textit{wsgi.py} se v základu skládá z jednoduché \zk{WSGI} konfigurace, již je možno podle potřeby dále upravovat.

\subsubsection{manage.py}
% http://www.moreware.org/books/The%20Definitive%20Guide%20to%20Django.pdf
% https://docs.djangoproject.com/en/2.2/topics/migrations/
Nástroj pro příkazový řádek \textit{manage.py} umožňuje spravovat Django projekt. Tento soubor není po vytvoření nijak upravován.

Pro vývoj aplikace lze použít odlehčený webový server Djanga, na němž může autor okamžitě začít budovat aplikaci, aniž by byla vyžadována konfigurace produkčního serveru. Vývojový server provádí kontrolu kódu a automaticky se po každé uložené změně znovu načte, bez nutnosti restartu. Tento server se spouští příkazem:

\begin{center}
\texttt{python manage.py runserver 0:8080}
\end{center}

Standardně se vývojový server spouští na interní \zk{IP} adrese a portu 8000. V případě, že je třeba zobrazovat webové stránky mimo stroj, na němž server běží, lze nastavit viditelnost a odlišný port serveru přidáním parametru \texttt{0:8080}. V takové situaci je stránka dostupná odkudkoliv při zadání adresy nastavené v proměnné \texttt{ALLOWED\_HOSTS} v souboru \textit{settings.py} a zvoleného portu.

Pro práci s databází jsou nezbytné dva základní příkazy:

\begin{center}
\texttt{python manage.py makemigrations}
\end{center}

který vytváří jednotlivé migrační soubory založené na změnách provedených v modelech Djanga a 

\begin{center}
\texttt{python manage.py migrate}
\end{center}

který tyto změny aplikuje do databáze. V případě, že databáze ještě neexistuje, tak ji automaticky vytvoří.

Migrace v Djangu funguje podobně jako verzovací systémy - \texttt{makemigrations} odpovídá příkazu \texttt{commit} a \texttt{migrate} pak obdobně jako \texttt{push} tyto změny propíše do databáze.

Poslední z významných funkcí \textit{manage.py} je spouštění jednotkových testů.

\subsubsection{db.sqlite3}
Výchozí databází v Djangu je relační databáze SQLite. Obsahuje informace o jednotlivých modelech a vztazích mezi nimi.

% https://docs.djangoproject.com/en/2.2/topics/migrations/#sqlite
SQLite nemá vhodně implementovánu podporu změn, Django ji proto nahrazuje postupem, kdy vytvoří novou tabulku s novým schématem, data z původní tabulky překopíruje do nové, starou smaže a novou přejmenuje podle prvotní. Proto se nedoporučuje tuto databázi používat v produkci, obzvlášť v případě většího množství dat a častých změn.

Kromě SQLite Django oficiálně podporuje databáze PostgreSQL, MySQL a Oracle.

\subsection{Webové aplikace}
Do projektu lze přidávat webové aplikace, jedna aplikace může být součástí více projektů. Pro její vytvoření lze užít konzolový nástroj:

\begin{center}
\texttt{python manage.py startapp nazev\_aplikace}
\end{center}

Implicitní struktura aplikace je vždy stejná:

\begin{itemize}
\item \texttt{\_\_init\_\_.py} - určuje aplikaci jako balíček Pythonu
\item \texttt{admin.py} - umožňuje registraci datových modelů
\item \texttt{apps.py} - definuje název aplikace??
\item \texttt{models.py} - umožňuje tvorbu datových modelů
\item \texttt{tests.py} - umožňuje tvorbu jednotkových testů
\item \texttt{views.py} - umožňuje definování pohledových funkcí
\end{itemize}

Každou aplikaci, která má být součástí projektu, je třeba zaregistrovat v souboru \textit{settings.py} (viz \ref{settings}) v položce \texttt{INSTALLED\_APPS}.

\section{Docker}
\label{docker}

% kouknout na licenci (můžu používat??)
\begin{figure}[H] \centering
      \includegraphics[width=150pt]{./pictures/Docker_(container_engine)_logo.png}
      \caption[Docker logo]{Docker logo (zdroj:
\href{https://commons.wikimedia.org/wiki/File:Docker_(container_engine)_logo.png}{Wikimedia Commons})}
      \label{fig:docker}
  \end{figure}

