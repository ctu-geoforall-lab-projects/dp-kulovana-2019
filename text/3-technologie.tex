\chapter{Použité technologie}
\label{3-technologie}

Třetí kapitola stručně představuje jednotlivé technologie použité při
tvorbě webového administrátorského rozhraní.

\section{LDAP}
Lightweight Directory Access Protocol neboli \zk{LDAP} je otevřený,
standardizovaný protokol pro~přístup k~adresářovým službám. Slouží k~zobrazování 
dat, jejich úpravám a ukládání na~adresářovém serveru přes~Internet 
Protocol (\zk{IP}). \cite{ldap}

Adresářová služba (directory service) je v~podstatě specializovaná
databáze, která slouží především k~procházení a vyhledávání, ke~změnám
dochází jen zřídka. Obvykle neposkytuje komplikovanější databázové
techniky, jako jsou transakce a operace nutné pro~zachování datové
integrity. Pro~zvýšení rychlosti odezvy mohou dokonce obsahovat i
duplicitní záznamy.

\zk{LDAP} je založen na~modelu server-klient. Jeden či více \zk{LDAP}
serverů obsahují data ve~stromové struktuře (directory information
tree, DIT). Klient požádá o~záznam odpovídající jeho dotazu. Odpověď
dostane buď ze~serveru, ke~kterému je připojen, nebo je odkázán na~jiný 
server, kde je informace uložena. Výsledek hledání bude vždy
stejný, ať už se připojí ke~kterémukoliv serveru. Tento rys je
důležitý obzvláště u~globálních adresářových služeb. \cite{openldap}

\zk{LDAP} model se skládá z~jednotlivých záznamů (entries). Záznam
sestává z~kolekce atributů (povinných a nepovinných) a má přiřazen
globálně unikátní identifikátor, tzv. Distinguished Name (\zk{DN}),
který je založen na~lokaci záznamu v~rámci hierarchického (stromového)
uspořádání modelu. \zk{DN} začíná vlastním jménem záznamu (Relative
Distinguished Name) a k~němu jsou jako řetěz připojena jména všech
jeho předků. Každý atribut má vlastní datový typ a může obsahovat
jednu či více hodnot. \cite{ldap}

Podle~uvedené ukázky stromové struktury by jednoznačným pojmenováním
pro~uživatele se jménem \textsf{uid=test01} bylo \zk{DN}
\textsf{uid=test01, ou=People, dc=org, dc=com}.

\begin{figure}[H] \centering
      \includegraphics[width=380pt]{./pictures/ldap_dit.png}
      \caption[Příklad stromové struktury LDAP]{Příklad stromové struktury LDAP (zdroj:
	  \href{}{Tereza Kulovaná})}
      \label{fig:ldap-dit}
\end{figure}

\zk{LDAP} poskytuje ochranu informací, které jsou na~serveru uloženy,
pomocí mecha\-nismu autentizace, který požaduje po~klientovi prokázaní a
ověření identity před~zobrazením jakýchkoliv dat.

Server \zk{LDAP} je vhodný pro~autentizaci uživatelů a klientských
zařízení, pro~sprá\-vu uživatelských informací, apod. Výhodou je, že
uživatelé potřebují pro~přístup k~různým aplikacím v~síti (pošta, ftp)
pouze jedny přihlašovací údaje. Je nezávislý na~operačním systému.

Protokol LDAP vychází ze~standardu X.500, jehož je odlehčenou
variantou. Někdy je nazýván
X.500-lite.\footnote{https://www.webopedia.com/TERM/L/LDAP.html}

\newpage
\subsection{OpenLDAP}
\label{openldap}

\begin{figure}[H] \centering
      \includegraphics[width=100pt]{./pictures/LDAPlogo.png}
      \caption[OpenLDAP logo]{OpenLDAP logo (zdroj:
	  \href{http://www.openldap.org/images/headers/LDAPlogo.gif}{OpenLDAP.org})}
      \label{fig:ldap}
\end{figure}

OpenLDAP je otevřená (open-source) implementace protokolu \zk{LDAP}
vyvíjená \linebreak pod~hlavičkou OpenLDAP Project. OpenLDAP je distribuován pod~vlastní 
licencí OpenLDAP Public
License.\footnote{http://www.openldap.org/software/release/license.html}
Vývoj OpenLDAP má počátek v~roce 1998 a navazuje na~předešlou činnost
University of Michigan. \cite{openldap}

Hlavní komponenty:
\begin{itemize}
\item slapd - nezávislý \zk{LDAP} server
\item knihovny implementující \zk{LDAP} protokol
\item klientské nástroje (ldapsearch, ldapadd, ldapmodify,...)
\end{itemize}

\textit{slapd} (Standalone \zk{LDAP} daemon) je nezávislý \zk{LDAP}
adresářový server, který zachytává \zk{LDAP}připojení a odpovídá na~operace, 
které přes~tato spojení obdrží. Umožňuje připojení ke~globální 
\zk{LDAP} adresářové službě nebo může lokální služby
obsluhovat sám. Může běžet na~mnoha různých platformách.

Podporuje silnou autentizaci a bezpečnost dat díky metodě SASL (Simple
Authentication and Security Layer) a protokolu TLS (Transport Layer
Security). Poskytuje široké možnosti kontroly přístupu k~informacím v~databázi - 
přístup může být udělen mj. na~základě přihlašovacích
údajů, \zk{IP} adresy nebo \zk{DN}. \textit{slapd} nabízí celou škálu
databázových backendů - MDB, hierarchický, vysoce výkonný transakční
databázový backend; SHELL, backend pro~shellové skripty; a jednoduchý
backend PASSWD a jiné. \cite{openldap}

Konfigurace \textit{slapd} probíhá přes~konfigurační soubor, příklad takové změny je uveden v~kapitole \ref{web-console}. 

%ldapmodify, ldapadd, ldapsearch, ldapdelete?

GIS.lab využívá OpenLDAP 2.4. V~rámci webové aplikace probíhá
synchronizace s~Djangem, z~adresářové struktury OpenLDAP jsou
konkrétně využívány organizační jednotky \textit{People} a
\textit{Groups}. Náhled struktury GIS.labu je v~ukázce omezen jen na~záznamy 
potřebné k~vytvoření \zk{DN} těchto organizačních jednotek,
které v~realitě obsahují mnohem více záznamů.

\begin{figure}[H] \centering
      \includegraphics[width=400pt]{./pictures/gislab-ldap_dit.png}
      \caption[Ukázka stromové struktury GIS.labu]{Ukázka stromové struktury GIS.labu (zdroj:
	  \href{}{Tereza Kulovaná})}
      \label{fig:gislab-dit}
\end{figure}

\subsection{django-python3-ldap}
Knihovna \textit{django-python3-ldap} zajišťuje autentizaci uživatelů
s~\zk{LDAP} serverem a synchronizaci \zk{LDAP} uživatelů s~lokální
databází Djanga. Podporuje vlastní modely Djanga upravené specifickým
potřebám. Ve~výchozím nastavení je nakonfi\-gurována podpora přihlášení
přes~OpenLDAP, po~úpravě nastavení umožňuje připojení k~adresářové
službě Microsoft Active Directory. Vychází z~knihovny \textit{ldap3} a 
funguje jak pro~Python 2, tak pro~Python 3. \cite{django-python3}

Při~prvním přihlášení uživatele se aplikace pokusí vytvořit připojení
k~\zk{LDAP} serveru skrze poskytnuté uživatelské jméno a heslo. V~případě 
úspěšného připojení jsou údaje z~\zk{LDAP} serveru uloženy do~databáze 
Djanga a při~každém dalším přihlášení jsou aktualizovány.

Synchronizaci všech uživatelů zároveň zajišťuje příkaz
\textsf{manage.py ldap\_sync\_users}. Tuto akci lze provést
jednorázově, pro~pravidelnou synchronizaci může být automatizovaně
spouštěn na~pozadí skrze~softwarového démona Cron. \cite{django-python3}

Použití této knihovny je relativně snadné. Po instalaci stačí v~konfiguračním 
souboru Djanga \textit{settings.py} nastavit potřebné
proměnné (např. \zk{URL} adresu \zk{LDAP} serveru) a při~příštím
spuštění serveru již synchronizace funguje.

\subsection{ldap3}
Knihovna \textit{ldap3} je založena na~aplikaci protokolu \zk{LDAP}
v3. Poskytuje operace potřebné pro~připojení k~\zk{LDAP} serveru, pro~vyhledání 
záznamů, jejich vytvoření, úpravu a odstranění. Podporuje
verze Python 2 i 3. \cite{ldap3}


\section{Python}

\begin{figure}[H] \centering
      \includegraphics[width=150pt]{./pictures/python-logo-master-v3-TM.png}
      \caption[Python logo]{Python logo (zdroj:
\href{https://www.python.org/static/community_logos/python-logo-master-v3-TM.png}{Python.org})}
      \label{fig:python}
  \end{figure}
  
\uv{\textit{Python je vysokoúrovňový, interpretovaný programovací jazyk. Podporuje
procedurálně i objektově orientované programování, je výkonný, zároveň
má velmi jednoduchou a čistou syntax. V ostatních jazycích je
odsazování řádků doporučeno z~hlediska přehlednosti, u~Pythonu je
základním stavebním kamenem a je povinné.} \cite{diveintopython}

\textit{Dnes je Python vyvíjen jako open source projekt
pod~záštitou neziskové organizace Python Software Foundation
(PSF). Je distribuován pod~licencí PSF, která je
kompatibilní s~GPL. Je možné ho nainstalovat na~běžné platformy
jako Windows, Unix nebo Mac OS, pro~Linux je většinou součástí
základní instalace. Při~vyvarování se systémově závislých funkcí je
přenositelný mezi~platformami bez~jakýchkoli změn.}

\textit{Python má široké využití, od~jednoduchých programů po~rozsáhlé
aplikace. Právě pro~tyto možnosti, univerzálnost, přehlednost kódu a
výkonnost z~něj udělaly programovací jazyk, který je mezi~začátečníky ve
velké oblibě. Během krátké doby v~něm funkční skript zvládne napsat
každý.}} \citep{kulovana-bp}

Python v~současnosti existuje ve~dvou hlavních verzích - Python 2 a
Python 3. Python 3.0 byl vydán v~roce
2008\footnote{https://www.python.org/downloads/} a není zpětně
kompatibilní s~verzí Python 2. Hlavním důvodem pro~takto zásadní změnu
bylo rozhodnutí Guido van Rossuma, zakladatele jazyku Python, očistit
Python 2.x od~mnoha problémů pořádně v~jednom kroku. \cite{python-wiki}

Největšími změnami jsou upravení některých tříd a oddělení abstrakcí
\textsf{řetězec} a \textsf{posloupnost bytů}. Textový řetězec
(\textsf{str}) je nově převeden ve~výchozím nastavení na~typ unicode,
což vytváří konzistentnější a spolehlivější prostředí. Změny doznalo
chování operátoru dělení \textsf{/} celých čísel. Výsledkem je číslo s~plovoucí 
desetinnou čárkou, na~rozdíl od~předchozí verze, která
vracela celé číslo (ve verzi 3.x dostupné pod~operátorem
\textsf{//}). Mezi~nejviditelnější novinky pro~běžného uživatele je
přechod od~příkazu \textsf{print} k~funkci \textsf{print()}. \cite{python-notes}

Python 3 je obecně přívětivější k~učení nových uživatelů a je
považován za budoucnost tohoto jazyka. Stále však existuje mnoho
programů, jež fungují na~poslední verzi Python 2.7 vydané v~roce
2010\footnote{https://www.python.org/downloads/}, která již nedostává
žádné velké aktualizace. Důvody jsou různé - daný projekt byl vyvíjen
v Pythonu 2 a nejsou dostatečné kapacity na~jeho přechod na~novější
verzi; na~některých operačních systémech není Python 3 nainstalovaný a
uživatelé nemají vždy práva si ho sami doinstalovat; \linebreak existuje potřeba
využívat externí knihovnu, která podporuje pouze Python 2 a není
triviální ji převést do~Pythonu 3.

Poslední vydanou stabilní verzí je Python
3.7.\footnote{https://www.python.org, květen 2019}

\section{Django}

\begin{figure}[H] \centering
      \includegraphics[width=150pt]{./pictures/django-logo-positive.png}
      \caption[Django logo]{Django logo (zdroj:
\href{https://static.djangoproject.com/img/logos/django-logo-positive.png}{Djangoproject.com})}
      \label{fig:django}
  \end{figure}

Django je vysokoúrovňový webový framework napsaný v~jazyce Python. Je
udržovaný organizací Django Software Foundation (DSF). Djagno je bezplatné a
vydané pod~open-source licencí BSD. Název získalo po~jazzovém
kytaristovi Djangovi Reinhardtovi. \cite{definitiveguide}

Hlavním cílem Djanga je usnadnit tvorbu komplexních, databází řízených
webových aplikací. Pro~tento účel se řídí zásadou oddělení
zodpovědností (angl. Separation of concerns) a volně navazuje na~architekturu 
Model-view-controller (\zk{MVC}). \cite{definitiveguide} 
\newpage
\zk{MVC} architektura sestává ze~tří volně propojených komponent:
\begin{itemize}
\item model (model) - reprezentace dat, k~nimž aplikace přistupuje
\item view (pohled) - uživatelské rozhraní
\item controller (řadič) - reakce na~žádosti a zajištění změn v~pohledu nebo v~modelu
\end{itemize}

Poslední vydanou stabilní verzí v~době zpracování bylo Django 2.1 a
veškerý další popis uvedený níže platí pro~tuto verzi.

Frameworky mají snahu co největší množství práce automatizovat a tak
při~vytvo\-ření projektu pomocí příkazu:

\begin{center}
\textsf{django-admin startproject nazev\_projektu}
\end{center}

vznikne automaticky jeho základní kostra, která slouží jako podstata
již fungující Django aplikace. \cite{django-doc}

\dirtree{%
.1 project.			
	.2 mysite.
		.3 \_\_init\_\_.py.
		.3 settings.py.
		.3 urls.py.
		.3 wsgi.py.
	.2 db.sqlite3.
	.2 manage.py.
}

\subsubsection{\_\_init\_\_.py}
Soubor \textit{\_\_init\_\_.py} dává Pythonu najevo, že s~adresářem, v~němž 
se soubor nachází, má být zacházeno jako s~balíčkem modulů
Pythonu. Jedná se o~prázdný soubor, který se obvykle nijak nemění.

\subsubsection{settings.py}
\label{settings}
Soubor \textit{settings.py} skrývá nastavení Django projektu,
např. jakým způsobem má probíhat autentizace, kde jsou umístěné další
potřebné soubory či informace o~použi\-tých databázích a registrovaných
aplikacích.

\subsubsection{urls.py}
\textit{urls.py} obsahuje \zk{URL} cesty pro~vytvořený Django projekt,
v~podstatě se jedná o~jakýsi rejstřík stránky. Implicitně obsahuje
cestu k~vestavěné administrátorské konzoli. Při~propojení s~Django
aplikacemi jsou sem přidány další \zk{URL} adresy.

\subsubsection{wsgi.py}
\zk{WSGI} je specifikace popisující komunikaci mezi~webovým serverem a
webovou aplikací nebo frameworkem v~jazyce Python. Jedná se o~primární
nástroj nasazení programů v~Djangu. Soubor \textit{wsgi.py} se v~základu 
skládá z~jednoduché \zk{WSGI} konfigurace, již je možno podle~potřeby 
dále upravovat.

\subsubsection{manage.py}
Nástroj pro~příkazový řádek \textit{manage.py} umožňuje spravovat
Django projekt. Tento soubor není po~vytvoření nijak upravován.

Pro~vývoj aplikace lze použít odlehčený webový server Djanga, na~němž
může autor okamžitě začít budovat aplikaci, aniž by byla vyžadována
konfigurace produkčního serveru. Vývojový server provádí kontrolu kódu
a automaticky se po~každé uložené změně znovu načte, bez~nutnosti
restartu. \cite{django-doc} Tento server se spouští příkazem:

\begin{center}
\textsf{python manage.py runserver 0:8080}
\end{center}

Standardně se vývojový server spouští na~interní \zk{IP} adrese a
portu 8000. V~případě, že je třeba zobrazovat webové stránky mimo
stroj, na~němž server běží, lze nastavit viditelnost a odlišný port
serveru přidáním parametru \textsf{0:8080}. V~takové situaci je
stránka dostupná odkudkoliv při~zadání adresy nastavené v~proměnné
\textsf{ALLOWED\_HOSTS} v~souboru \textit{settings.py} a zvoleného
portu.

Pro~práci s~databází jsou nezbytné dva základní příkazy:

\begin{center}
\textsf{python manage.py makemigrations}
\end{center}

který vytváří jednotlivé migrační soubory založené na~změnách
provedených v~modelech Djanga a

\begin{center}
\textsf{python manage.py migrate}
\end{center}

který tyto změny aplikuje do~databáze. V případě, že databáze ještě
neexistuje, tak ji automaticky vytvoří. \cite{django-doc}

Migrace v~Djangu funguje podobně jako verzovací systémy -
\textsf{makemigrations} odpovídá příkazu \textsf{commit} a
\textsf{migrate} pak obdobně jako \textsf{push} tyto změny propíše do~databáze.

Poslední z~významných funkcí \textit{manage.py} je spouštění
jednotkových testů.

\subsubsection{db.sqlite3}
Výchozí databází v~Djangu je relační databáze SQLite. \cite{django-doc} 
Obsahuje informace o~jednotlivých modelech a vztazích mezi~nimi.

SQLite nemá vhodně implementovánu podporu změn, Django ji proto
nahrazuje postupem, kdy vytvoří novou tabulku s~novým schématem, data
z~původní tabulky překopíruje do~nové, starou smaže a novou přejmenuje
podle~prvotní. Proto se nedoporučuje tuto databázi používat v~produkci, 
obzvlášť v~případě většího množství dat a častých změn.

Kromě SQLite Django oficiálně podporuje databáze PostgreSQL, MySQL a
\linebreak Oracle. \cite{django-doc}

\subsection{Webové aplikace}
\label{django-app}
Do~projektu lze přidávat webové aplikace, jedna aplikace může být
součástí více projektů. Pro~její vytvoření lze užít konzolový nástroj:

\begin{center}
\textsf{python manage.py startapp nazev\_aplikace}
\end{center}

Implicitní struktura aplikace je vždy stejná:

\begin{itemize}
\item \textsf{\_\_init\_\_.py} - určuje aplikaci jako balíček Pythonu
\item \textsf{admin.py} - umožňuje registraci datových modelů
\item \textsf{apps.py} - definuje název aplikace
\item \textsf{models.py} - umožňuje tvorbu datových modelů
\item \textsf{tests.py} - umožňuje tvorbu jednotkových testů
\item \textsf{views.py} - umožňuje definování pohledových funkcí
\end{itemize}

Každou aplikaci, která má být součástí projektu, je třeba
zaregistrovat v~souboru \textit{settings.py} (viz \ref{settings}) v~položce 
\textsf{INSTALLED\_APPS}.

\newpage
\section{Docker}
\label{docker}

\begin{figure}[H] \centering
      \includegraphics[width=150pt]{./pictures/Docker_(container_engine)_logo.png}
      \caption[Docker logo]{Docker logo (zdroj:
\href{https://commons.wikimedia.org/wiki/File:Docker_(container_engine)_logo.png}{Wikimedia Commons})}
      \label{fig:docker}
  \end{figure}
  
Docker je platforma pro~vývoj, nasazení a běh aplikací. Poskytuje
jednotné rozhraní pro~izolaci procesů do~standardizovaných balíčků,
tzv. kontejnerů. Kontej\-nery jsou tvořeny vlastním softwarem,
knihovnami a konfiguračními soubory. Jsou na~sobě navzájem nezávislé,
ale mohou mezi~sebou komunikovat skrze~definované kanály. Na~rozdíl 
od~virtuálních strojů neobsahují kontejnery operační systém, ale sdílejí
kernel (jádro \zk{OS}\footnote{\uv{Jádro operačního systému je v~informatice část operačního systému, která je zavedena do~operační paměti při~startu počítače a je jí předáno řízení.}(\href{https://cs.wikipedia.org/wiki/Jádro_operačního_systému}{Wikipedie})}) s~hlavním operačním systémem. Díky tomu je režie při~jejich
spuštění výrazně nižší a mají mnohem menší velikost. \cite{docker-book}

Kontejnery jsou vytvořeny z~tzv. obrazů (images). Image je spustitelný
balíček, který obsahuje všechno potřebné pro~běh aplikace (kód,
knihovny, proměnné pro\-středí, konfigurační soubory). Využívá se ke~skladování 
a sdílení aplikací. Z~jednoho obrazu je možné spustit
jakékoliv množství kontejnerů. \cite{docker-doc}

Docker existuje v~placené formě i jako open-source software. Funguje 
v~prostředí Linuxu, novějších verzích Windows a ve~vybraných cloudových
službách. \cite{docker-doc} Skládá se ze~tří komponent:

\begin{itemize}
\item software - Docker démon, proces, který řídí Docker kontejnery
\item objekty - entity sloužící k~sestavení aplikace v~Dockeru (např. obrazy, kontejnery, služby)
\item registr - repozitář Docker obrazů
\end{itemize}

\newpage
\section{Ansible}
\label{ansible}  
Ansible je jednoduchý automatizační nástroj, který umožňuje
konfiguraci systémů, víceuzlové nasazení aplikací a orchestraci jiných
pokročilejších úloh. Orchestrace je řízena z~jednoho řídícího stroje,
jenž přistupuje ke~spravovaným uzlům přes~SSH nebo PowerShell. Ansible
využívá architekturu, jež běží bez~agentů, tedy na~uzly jsou jen
dočasně nainstalovány a spuštěny tzv. moduly pomocí SSH. Ve~chvíli,
kdy Ansible uzly neřídí, uzel nespotřebovává žádné prostředky, jelikož
na~něm není nainstalován démon, který by sám software spouštěl. \cite{ansible-doc}

K~zápisu znovupoužitelného popisu stavu uzlů jsou vytvářeny Ansible
Playbooks. Jedná se o~soubory psané v~jazyce YAML, který se vyznačuje
pro~člověka snadnou čitelností.
