\chapter*{Závěr}
\addcontentsline{toc}{chapter}{Závěr}
\markboth{ZÁVĚR}{} %zajisti, aby byl text uvod v~zahlavi
\label{5-zaver}

% Popsat, které knihovny byly nakonec zvoleny??

Tato diplomová práce si kladla za cíl vytvořit webové administrátorské rozhraní pro potřeby platformy GIS.lab. Zadání se podařilo splnit v primární, neodladěné formě. Během zpracování se vyskytlo několik dalších požadavků, které bude třeba do finální podoby zakomponovat.

V době odevzdání práce obsahuje administrátorská konzole základní funkcionalitu: přidávání nových uživatelů, jejich úpravu a odebírání, správu příslušnosti k rolím (skupinám), vytváření a mazání těchto rolí. Pro běžného uživatele je prozatím dostupná registrace, přihlášení pod svým účtem do uživatelské konzole, která zobrazuje osobní informace a role, jichž je uživatel členem. Své osobní informace, včetně hesla, může uživatel měnit. 

Komunikace mezi webovou aplikací a LDAP serverem funguje obousměrně, ale vyžaduje do budoucna ještě odladění, doplnění některých okrajových případů a v první řadě zajištění pravidelné automatické synchronizace. Nejzásadnější problematické situace v aktuální verzi nastávají v případě odstranění uživatele na LDAP serveru - změna se nyní okamžitě neprojeví v databázi webové aplikace a pokud byl uživatel v momentu smazání přihlášen, může i nadále upravovat svoje údaje.

Role může aktuálně uživateli přiřazovat jen administrátor, záměrem je umožnit uživateli vybrat si potřebné role už při registračním procesu, případně následně zažádat o změnu přes uživatelskou konzoli.

V současnosti jsou všechny změny propisovány do databáze okamžitě. Cílovým stavem, po vyplnění registračního formuláře a ověření emailové adresy uživatele, je poslat emailové upozornění administrátorovi s žádostí o vytvoření nového účtu. Teprve po potvrzení správcem funkční uživatelský účet skutečně vznikne. Stejným potvrzovacím procesem budou procházet i žádosti o přiřazení nové role.

Důležitým prvkem, který bude třeba přidat, je ověřování platnosti emailové adresy během procesu vytváření nového uživatele i při případné následné změně emailu samotným uživatelem. Pro lepší přehlednost budou pro uživatele také přidány informativní zprávy o úspěšné změně osobních údajů či hesla.

Administrátorská konzole je ponechána v původním designu Djanga, design uživatelské konzole byl inspirován vzhledem webové platformy Gisquick, avšak v konečné verzi dozná ještě dalších úprav.

Podstatným bodem z hlediska vývoje je zprovoznění plnohodnotného výpisu logů a doplnění dalších informativních zpráv o probíhajících procesech ve webové konzoli. Veškerá současná řešení i finální podobu bude třeba řádně ověřit pomocí automatizovaných testů.

Otázkou, na níž bude třeba teprve najít odpověď, je, jakým způsob bude vhodné se postavit ke schopnosti administrátorů měnit osobní údaje a případně i heslo dalších administrátorů. Nyní mají všichni správci přístup k údajům všech ostatních osob v databázi a u každé z nich mohou upravovat cokoliv. Bezpečnost je dalším prvkem, který bude vyžadovat hlubší prozkoumání, především s ohledem na předávání hesla mezi webovou aplikací a LDAP serverem a také následné šifrování hesla na LDAP serveru. 

V roce 2015 vznikl prvotní návrh Python knihovny, která by měla nahradit stávající shellové skripty při tvorbě uživatelů (více viz kapitola \ref{python-knihovna}). Její vývoj byl na jistou dobu pozastaven, ale jedním z ambicioznějších cílů této práce bylo její dokončení a propojení s webovou aplikací. To se před termínem odevzdání nepodařilo, ale práce na knihovně byly obnoveny a v krátké době snad budou úspěšně završeny.

% Doplnit jak bude probíhat integrace přes Docker.
Integrace do stávající architektury platformy GIS.lab proběhne přes Docker kontejner. Při instalaci platformy GIS.lab si bude moci uživatel zvolit, zda si přeje mít dostupné i webové administrační rozhraní či ne. Pokud ano, z konfiguračních souborů vznikne Docker obraz, na jehož základě bude vytvořen kontejner. Konzole, včetně webového serveru, poběží v izolovaném prostředí. Pro zápis konfigurace bude využit Ansible playbook.

Webové administrátorské a uživatelské rozhraní tedy nyní obsahuje základní funkcionalitu s tím, že bude snaha je v brzké době dokončit a zařadit mezi služby platformy GIS.lab.