\chapter{Úvod}
\label{1-uvod}

% http://gislab-npo.github.io/gislab/index.html

% ML: GISová bych vynechal (detail)
Open-source GISová platforma GIS.lab slouží k rychlému a jednoduchému
nasazení centrálně řízené GIS infrastruktury v lokální síti (LAN),
% skloňování
data centra nebo cloudové služby. Poskytuje celistvý soubor
% ML: neplaceny neni zcela podstatne, ale melo by zaznit (poznamka)
neplaceného GIS softwaru integrovaného do jednoho systému, který je
okamžitě připraven k použití.
% ML: nevim, zda to je popsano dale, ale chtelo by nacrtnout rozsah
% softwarovych komponent od DB pres analyticke nastroje az po webove
% sluzby a publikacni platformu (Gisquick)

% ML: spravce systemu
Všechny použité technologie jsou plně pod kontrolou správců, náklady
na nasazení a vlastnictví takovéhoto komplexního řešení jsou sníženy
na absolutní minimum. Díky tomu je možné GIS.lab využít v oblastech a
% ML nasazeni proprietarnich techno...
podmínkách, kde by aplikace jiné technologie nebyla cenově dostupná či
%% ML: ty zony zasazene prirodni katastrofou bych vynechal (to je asi
%% prevzato z dokumentace), spis bych pridal rozvojove zeme
technicky možná. Příkladem mohou být zóny zasažené přírodní
katastrofou či sféra školství a vzdělávacích institucí.

% Co to už umí?

% ML: mobilniho klienta nezminuj, ten se dal nevyviji. Mluvit o
% Gisquicku jako o webovem klientovi GIS.labu take neni presne, pote
% co se osamostatnil. Gisquick jako publikacni platforma muze byt
% zpetne integrovana do GIS.labu...
GIS.lab je dostupný jako desktopový, webový a mobilní klient. Webový a
mobilní klient jsou přístupné díky samostatné platformě Gisquick,
která je automaticky integrovaná i do desktopové verze. GIS.lab
% ML: nejsirsi skalu
Desktop obsahuje nejvíce funkcionalit, z nichž mezi nejdůležitější
% ML: lze ukladam v souborovem systemu anebo v PostGIS databazi
patří ukládání prostorově i neprostorově orientovaných dat a jejich
% ML: sklonovani
sdílení, tvorba a analýza vektorových, rastrových i tabulkových dat
nebo rychlé vytváření kartografických výstupů.

% Co to má ještě do budoucna umět?
% ML: ono jde v podstate o umozneni pristupu k jednotlivym komponentam mimo desktopoveho klienta
Vývoj GIS.labu ještě ani zdaleka není u konce. Nabízené portfolio se
má do budoucna dělit na čtyři základní služby - přístup k PostGIS
databázi, publikaci dat pomocí webových mapových služeb (\zk{WMS},
\zk{WFS}), webovou aplikaci a desktopového klienta (více viz kapitola
\ref{vision}).
% ML: Do budoucna si dokazu predstavit dalsi sluzby typu vypocetniho serveru na bazi WPS a pod.

%Pro snazší administraci uživatelů a jejich přístupových práv ke zmiňovaným službám bylo rozhodnuto vytvořit webové administrátorské rozhraní. Jednodušší přístup  

% Důvod pro moji DP.
Vývojáři GIS.labu se obecně snaží o co nejvíce uživatelsky přívětivé
prostředí, což vedlo k rozhodnutí (potřebě) vytvořit webové
administrátorské rozhraní pro snazší správu uživatelů a definování
jejich přístupových práv ke zmiňovaným službám. Současný systém
administrace přes příkazovou řádku není pro některé správce
srozumitelný, navíc neumožňuje žádosti o registraci a o přiřazení
přístupových práv přímo ze strany uživatelů. Pro naplnění tohoto
požadavku vzniklo zadání této diplomová práce.

% Volba technologií
% ML: kombinace budouciho a minuleho casu
Při návrhu webové aplikace bude vyvinuta snaha o její maximální
integraci do stávající architektury platformy GIS.lab. Pro vytvoření
webového rozhraní byl zvolen framework Django, který využívá, dnes již
od GIS.labu oddělená, platforma Gisquick. Důležitou vlastností tohoto
frameworku je i to, že je psaný v jazyce Python, což umožní navázání
na existující, ale nedokončenou knihovnu pro administraci uživatelů z
roku 2015, která by měla nahradit současné bashové skripty. Pro
ověření mezi webovou aplikací a LDAP serverem obsahujícím uživatelské
informace bude využita některá z dostupných externích knihoven
(např. django-python3-ldap, ldap3).
% jak nahradit pojem bashové skripty??
% ML: asi shellove (o moc lepsi to ale neni), https://cs.wikipedia.org/wiki/Shellov%C3%BD_skript

% Teoretická část
V teoretické části bude čtenář především podrobněji seznámen s
% ML: jiz zde bych naznacit k cemu vubec LDAP slouzi (v jedne kratke vete)
platformou GIS.lab, jejím budoucím rozšířením a protokolem LDAP.

Rešerše

jiné open source projekty, které taky používali nějakou webovou aplikaci pro správu uživatelů
