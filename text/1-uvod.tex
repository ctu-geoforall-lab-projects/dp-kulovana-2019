\chapter*{Úvod}
\addcontentsline{toc}{chapter}{Úvod}
\markboth{ÚVOD}{} %zajisti, aby byl text uvod v~zahlavi
\label{1-uvod}

% http://gislab-npo.github.io/gislab/index.html

Open-source platforma GIS.lab slouží k rychlému a jednoduchému
nasazení centrálně řízené GIS infrastruktury v lokální síti (LAN),
data centru nebo cloudové službě. Poskytuje celistvý soubor
neplaceného GIS softwaru integrovaného do jednoho systému, který je
okamžitě připraven k použití.
% ML: nevim, zda to je popsano dale, ale chtelo by nacrtnout rozsah
% softwarovych komponent od DB pres analyticke nastroje az po webove
% sluzby a publikacni platformu (Gisquick)

Všechny použité technologie jsou plně pod kontrolou správců systému, náklady
na nasazení a vlastnictví takovéhoto komplexního řešení jsou sníženy
na absolutní minimum. Díky tomu je možné GIS.lab využít v oblastech a
podmínkách, kde by aplikace proprietárních technologií nebyla cenově dostupná či
technicky možná. Příkladem mohou být rozvojové země či sféra školství a vzdělávacích institucí.

% Co to už umí?
GIS.lab je dostupný ve formě desktopového klienta. Vytváření webových aplikací je přístupné díky samostatné platformě Gisquick, která je automaticky integrovaná i do desktopové verze. GIS.lab
Desktop obsahuje širokou škálu funkcionalit, z nichž mezi nejdůležitější
patří ukládání prostorově i neprostorově orientovaných dat a jejich
sdílení, tvorba a analýza vektorových, rastrových i tabulkových dat
či rychlé vytváření kartografických výstupů. Data mohou být ukládána buď v souborovém systému, nebo v PostGIS databázi.

% Co to má ještě do budoucna umět?
Vývoj GIS.labu ještě ani zdaleka není u konce. Nabízené portfolio se
má do budoucna dělit na pět základních balíčků - datasety otevřených dat, přístup k PostGIS
databázi, publikaci dat pomocí webových mapových služeb (\zk{WMS},
\zk{WFS}), webovou aplikaci a desktopového klienta (více viz kapitola
\ref{vision}). Všechny komponenty by měly být dostupné jak společně, tak i samostatně, nezávisle na desktopovém klientovi. Mezi dalšími rozšířeními by mohl být například výpočetní server na bázi WPS (Web Processing Service) a další.

% Důvod pro moji DP.
Vývojáři GIS.labu se obecně snaží o co nejvíce uživatelsky přívětivé
prostředí, což vedlo k rozhodnutí (potřebě) vytvořit webové
administrátorské rozhraní pro snazší správu uživatelů a definování
jejich přístupových práv ke zmiňovaným službám. Současný systém
administrace přes příkazovou řádku není pro některé správce
srozumitelný, navíc neumožňuje žádosti o registraci a o přiřazení
přístupových práv přímo ze strany uživatelů. Pro naplnění tohoto
požadavku vzniklo zadání této diplomová práce.

%TODO tady možná zmínit, že se bude jednat o admin a uživatelskou konzoli

% Volba technologií
Při návrhu webové aplikace bude vyvinuta snaha o její maximální
integraci do stávající architektury platformy GIS.lab. Pro vytvoření
webového rozhraní bude zvolen framework Django, který využívá, dnes již
od GIS.labu oddělená, platforma Gisquick. Důležitou vlastností tohoto
frameworku je i to, že je psaný v jazyce Python, což umožní navázání
na existující, ale nedokončenou knihovnu pro administraci uživatelů z
roku 2015, která by měla nahradit současné shellové skripty. Pro
ověření mezi webovou aplikací a LDAP serverem obsahujícím uživatelské
informace bude využita některá z dostupných externích knihoven
(např. django-python3-ldap, ldap3).

% Teoretická část
V teoretické části bude čtenář především podrobněji seznámen s
platformou GIS.lab, jejím budoucím rozšířením a protokolem LDAP, 
který slouží k přístupu k datům, jejich úpravám a ukládání na 
adresářovém serveru.

\newpage
\subsection*{Rešerše}

Framework Django automaticky po instalaci obsahuje vestavěné
administrační roz\-hraní. To slouží ke správě záznamů v lokální
databázi. Django umožňuje využít připravené modely \textit{User} a
\textit{Group} nebo si vytvářet vlastní. Bylo vyvinuto jako komplexní
nástroj pro zobrazování a spravování článků, komentářů, uživatelů a
autorů v novinářském prostředí. Administrátorské rozhraní pro správu
uživatelů a jejich příslušnosti ke skupinám je jen malou podmnožinou
toho, co všechno tento framework umí. Django nabízí obsáhlou
dokumentaci \cite{django-doc}, která usnadňuje jak orientaci v
problematice, tak vlastní vývoj.

Prvotní inspirací pro webovou konzoli byla platforma Gisquick (viz
\ref{gisquick}), která využívá framework Django. Jedná se o webovou
mapovou aplikaci, do níž se mohou uživatelé zaregistrovat a po
přihlášení zveřejňovat a spravovat své projekty. Po oddělení od
GIS.labu jsou uživatelské účty uloženy ve výchozí databázi
SQLite. Verze, jež je integrovaná v GIS.labu, ale ověřuje přihlašovací
údaje vůči \zk{LDAP} serveru. Zvolené nastavení připojení k \k{LDAP}
bylo použito v počáteční fázi vývoje webové konzole GIS.lab.

Dalšími zajímavými projekty, které využívají Django pro správu
uživatelů je např. chatovací aplikace Zulip
(\href{https://zulipchat.com}{https://zulipchat.com}) nebo platforma
pro online prodej Oscar
(\href{http://oscarcommerce.com}{http://oscarcommerce.com}).

%TK: ještě rozšířím, když zbyde čas
