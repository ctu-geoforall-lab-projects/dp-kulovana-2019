\chapter{Úvod}
\label{1-uvod}

% http://gislab-npo.github.io/gislab/index.html

Open-source GISová platforma GIS.lab slouží k rychlému a jednoduchému nasazení centrálně řízené GIS infrastruktury v lokální síti (LAN), data centra nebo cloudové služby. Poskytuje celistvý soubor neplaceného GIS softwaru integrovaného do jednoho systému, který je okamžitě připraven k použití.
 
Všechny použité technologie jsou plně pod kontrolou správců, náklady na nasazení a vlastnictví takovéhoto komplexního řešení jsou sníženy na absolutní minimum. Díky tomu je možné GIS.lab využít v oblastech a podmínkách, kde by aplikace jiné technologie nebyla cenově dostupná či technicky možná. Příkladem mohou být zóny zasažené přírodní katastrofou či sféra školství a vzdělávacích institucí.

% Co to už umí?
GIS.lab je dostupný jako desktopový, webový a mobilní klient. Webový a mobilní klient jsou přístupné díky samostatné platformě Gisquick, která je automaticky integrovaná i do desktopové verze. GIS.lab Desktop obsahuje nejvíce funkcionalit, z nichž mezi nejdůležitější patří ukládání prostorově i neprostorově orientovaných dat a jejich sdílení, tvorba a analýza vektorových, rastrových i tabulkových dat nebo rychlé vytváření kartografických výstupů.

% Co to má ještě do budoucna umět?
Vývoj GIS.labu ještě ani zdaleka není u konce. Nabízené portfolio se má do budoucna dělit na čtyři základní služby - přístup k PostGIS databázi, publikaci dat pomocí webových mapových služeb (\zk{WMS}, \zk{WFS}), webovou aplikaci a desktopového klienta (více viz kapitola \ref{vision}). 

%Pro snazší administraci uživatelů a jejich přístupových práv ke zmiňovaným službám bylo rozhodnuto vytvořit webové administrátorské rozhraní. Jednodušší přístup  

% Důvod pro moji DP.
Vývojáři GIS.labu se obecně snaží o co nejvíce uživatelsky přívětivé prostředí, což vedlo k rozhodnutí (potřebě) vytvořit webové administrátorské rozhraní pro snazší správu uživatelů a definování jejich přístupových práv ke zmiňovaným službám. Současný systém administrace přes příkazovou řádku není pro některé správce srozumitelný, navíc neumožňuje žádosti o registraci a o přiřazení přístupových práv přímo ze strany uživatelů. Pro naplnění tohoto požadavku vzniklo zadání této diplomová práce.

% Volba technologií
Při návrhu webové aplikace bude vyvinuta snaha o její maximální integraci do stávající architektury platformy GIS.lab. Pro vytvoření webového rozhraní byl zvolen framework Django, který využívá, dnes již od GIS.labu oddělená, platforma Gisquick. Důležitou vlastností tohoto frameworku je i to, že je psaný v jazyce Python, což umožní navázání na existující, ale nedokončenou knihovnu pro administraci uživatelů z roku 2015, která by měla nahradit současné bashové skripty. Pro ověření mezi webovou aplikací a LDAP serverem obsahujícím uživatelské informace bude využita některá z dostupných externích knihoven (např. django-python3-ldap, ldap3). 
% jak nahradit pojem bashové skripty??

% Teoretická část
V teoretické části bude čtenář především podrobněji seznámen s platformou GIS.lab, jejím budoucím rozšířením a protokolem LDAP.

Rešerše

jiné open source projekty, které taky používali nějakou webovou aplikaci pro správu uživatelů