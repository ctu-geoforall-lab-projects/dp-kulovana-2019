\chapter{Rešerše}
\label{0-reserse}

Framework Django automaticky po~instalaci obsahuje vestavěné
administrační roz\-hraní. To slouží ke~správě záznamů v~lokální
databázi. Django umožňuje využít připravené modely \textit{User} a
\textit{Group} nebo si vytvářet vlastní. Bylo vyvinuto jako komplexní
nástroj pro~zobrazování a spravování článků, komentářů, uživatelů a
autorů v~novinářském prostředí. Administrátorské rozhraní pro~správu
uživatelů a jejich příslušnosti ke~skupinám je jen malou podmnožinou
toho, co všechno tento framework umí. Django nabízí obsáhlou
dokumentaci \cite{django-doc}, která usnadňuje jak orientaci v~
problematice, tak vlastní vývoj.

Prvotní inspirací pro~webovou konzoli byla platforma Gisquick (viz
\ref{gisquick}), která využívá framework Django. Jedná se o~webovou
mapovou aplikaci, do~níž se mohou uživatelé zaregistrovat a po~přihlášení 
zveřejňovat a spravovat své projekty. Po~od\-dělení od~GIS.labu 
jsou uživatelské účty uloženy ve~výchozí databázi
SQLite. Verze, jež je integrovaná v~GIS.labu, ale ověřuje přihlašovací
údaje vůči \zk{LDAP} serveru. Zvolené nastavení připojení k~\zk{LDAP}
bylo použito v~počáteční fázi vývoje webové konzole GIS.lab.

Dalšími zajímavými projekty, které využívají Django pro~správu
uživatelů je např. chatovací aplikace Zulip
(\href{https://zulipchat.com}{https://zulipchat.com}) nebo platforma
pro~online prodej Oscar
(\href{http://oscarcommerce.com}{http://oscarcommerce.com}).