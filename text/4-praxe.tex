\chapter{Administrátorské rozhraní}
\label{4-praxe}

\section{Současná správa uživatelských účtů}
\label{cmd-line}
% https://github.com/gislab-npo/gislab/tree/314fe436e1b65783d65e61000ca6d3f8ba873b2f/system/admin

Aktuálně probíhá správa uživatelských účtů spouštěním shellových
skriptů. Všechny příkazy je nutné spouštět pod právem
\textit{sudo} (jako administrátor). Některé příkazy lze spustit 
s parametry, které mohou být buď povinné, nebo nepovinné.

Seznam názvů dostupných skriptů pro GIS.lab administraci vrací příkaz
\textsf{gislab-help}. Každý jednotlivý příkaz z tohoto soupisu pak lze
spustit s příznakem \textsf{-h}, který zobrazí podrobnější
nápovědu. Uživatel se dozví, co skript vykoná a za jakých podmínek,
jakým způsobem příkaz zapsat do konzole a získá přehled parametrů,
které může či musí použít.

Níže jsou popsány skripty pro vytváření uživatelů a skupin, jejich
mazání a pro výpis existujících entit, protože souvisí s funkčností,
kterou nabízí nové webové rozhraní. Vedle toho existují i další,
např. pro zálohování dat nebo upgrade GIS.lab systému.

%TK: tady u těch popisů nevím, jestli dávat někam přesnou podobu příkazu (gislab-adduser [OPTIONS] username)
%přímo jako nadpis (\subsubsection{gislab-adduser...}) nebo někam pod něj? nebo to nepsat vůbec?
%nebo mít u každého konkrétní příkaz: $ sudo gislab-adduser -g User -l GIS.lab -m lab1@gis.lab -p lab lab1 ??

\subsubsection{gislab-adduser}
\begin{itemize}
\item [-g] křestní jméno (povinné)
\item [-l] příjmení (povinné)
\item [-m] email (povinné)
\item [-d] popis (nepovinné)
\item [-p] heslo (nepovinné)
\item [-s] přidat uživateli status administrátora (nepovinné)
\item [-a] přidat uživatele do vybraných skupin (nepovinné)
\end{itemize}
Tento příkaz má několik příznaků, z nichž část je povinná, část
nepovinná. Parametr \textsf{-p} musí být použit jako poslední, těsně
před názvem uživatele (username), může však být aplikován v různých
obměnách. První variantou je \textsf{-p PASSWORD}, která nastaví heslo
na hodnotu PASSWORD. Pokud je použit parametr bez argumentu
(\textsf{-p}), uživatel je následně dotázán na heslo. V třetím případě
je parametr z příkazu kompletně vynechán a heslo je automaticky
vygenerováno. Tedy i přestože parametr patří mezi nepovinné, heslo je
vždy po doběhnutí skriptu vytvořeno.

Uživatele lze přidat do více skupin zároveň použitím parametru
\textsf{-a} a seznamu skupin oddělených od sebe čárkami. Pokud je k
účtu přiřazen status administrátora (superuser), tak takový uživatel
může na klientských počítačích spouštět operace pod právem
\textit{sudo}.

\subsubsection{gislab-moduser}
\begin{itemize}
\item [-a] přidat uživatele do vybrané skupiny
\item [-A] odebrat uživatele z vybrané skupiny
\item [-s] přidat uživateli status administrátora
\item [-S] odebrat uživateli status administrátora
\item [-m] změnit email
\item [-p] změnit heslo
\item [-d] změnit popis
\end{itemize}
Upraví jeden či více atributů. Pokud nebyly předtím definovány, tak je
vytvoří. Stejně jako při vytváření uživatelského účtu je možné
přidávat a odebírat členství ve skupinách hromadně, jsou-li v seznamu
a oddělené čárkami. V případě, že je parametr \textsf{-p} uveden bez
argumentu, je heslo vygenerováno automaticky.

\subsubsection{gislab-deluser}
\begin{itemize}
\item [-b] zazálohovat uživatelská data (nepovinné)
\item [-f] vynutit proběhnutí tohoto příkazu (nepovinné)
\end{itemize}
Smaže uživatelský účet včetně příslušnosti ke skupinám. Pokud je
příkaz spuštěn s parametrem \textsf{-f}, proběhne vše okamžitě, v
opačném případě musí uživatel ještě jednou potvrdit, že si skutečně
přeje účet odstranit.

\subsubsection{gislab-addgroup}
\begin{itemize}
\item [-d] popis (nepovinné)
\end{itemize}
Vytvoří skupinu.

\subsubsection{gislab-delgroup}
\begin{itemize}
\item [-f] vynutit proběhnutí tohoto příkazu
\end{itemize}
Smaže skupinu, pokud je prázdná. Existují-li uživatelé, kteří patří do
mazané skupiny, vypíše se chybová hláška a skupina zůstane v
systému. Nejdříve je potřeba skupinu vyčistit přes
\textsf{gislab-moduser} a pak příkaz zopakovat.

Obdobně jako při mazání uživatelského účtu je možné vynutit proběhnutí
příkazu bez dalších dotazů. Pokud však zůstali nějací uživatelé ve
skupině, je vrácena stejná chyba, která byla popsána výše.

\subsubsection{gislab-listusers}
\begin{itemize}
\item [-g] vypsat pouze uživatele zvolené skupiny
\end{itemize}
Bez příznaku vypíše seznam všech uživatelů. U každého uživatele jsou
uvedeny všechny jeho atributy. S příznakem \textsf{-g nazev\_skupiny}
vypíše list uživatelů příslušících této skupině, včetně jejich
atributů. Heslo je zobrazeno v šifrované podobě.

\subsubsection{gislab-listgroups}
Kromě nápovědy nemá žádné parametry. Vypíše seznam všech skupin a
jejich atributů, včetně identifikátorů \textit{uid} uživatelů, kteří
do dané skupiny patří.

Jak \textsf{gislab-listusers}, tak \textsf{gislab-listgroups} vracejí
při větším množství záznamů velmi dlouhý seznam, který vypisují do
konzole. Zobrazit jen požadovanou část informací umožňuje program
grep.

\textsf{\$ sudo gislab-listusers | grep uid:}

\textsf{uid: uid=user01}

\textsf{uid: uid=user02}

%TK: u listusers a listgroups by možná pro lepší orientaci pomohlo přidat obrázky nebo textový výpis, jak vlastně vypadají výsledky, které vrací:
%dn: uid=gislab,ou=People,dc=gis,dc=lab                                                                                  objectClass: inetOrgPerson                                                                                              objectClass: posixAccount                                                                                               objectClass: shadowAccount                                                                                              uid: gislab                                                                                                             uidNumber: 3000                                                                                                         gidNumber: 3001                                                                                                         homeDirectory: /mnt/home/gislab                                                                                         loginShell: /bin/bash                                                                                                   cn: Administrator GIS.lab (gislab_vagrant_bionic)                                                                       sn: GIS.lab (gislab_vagrant_bionic)                                                                                     givenName: Administrator                                                                                                mail: gislab@gis.lab                                                                                                    userPassword:: e1NTSEF9Y0JBWUdMcVgxNTdweVJreXdxZzJRaUlpTE9CaHNaSTU=                                                     description: fjdsklfjdsl 

\section{Webové administrátorské a uživatelské rozhraní}
\label{web-console}
Technologie použité při vývoji byly zvoleny na základě těch již
aplikovaných v rámci platformy GIS.lab, resp. Gisquick. Pro zpracování
byl zvolen programovací jazyk Python 3. V první řadě se jedná o
budoucnost tohoto jazyka, na rozdíl od Pythonu 2. Také to umožní
navázat na koncept existující knihovny pro správu uživatelů z roku
2015. Navíc je to jazyk, v němž je napsán framework Django, který byl
použit také při tvorbě webové aplikace Gisquick. Ta byla původně
součástí GIS.labu a tímto způsobem zůstane celá široká základna v
jedné technologii.

Pro přístup k \zk{LDAP} serveru bylo třeba vyvíjet konzoli přímo v
GIS.labu. Ten běžel ve virtuálním vývojovém prostředí Vagrant, v
počátcích na virtuálním serveru Výpočetního a informačního centra ČVUT
(\href{gislab-vm.fsv.cvut.cz}{gislab-vm.fsv.cvut.cz}), později na
školním počítači
(\href{http://b802-01.fsv.cvut.cz}{http://b802-01.fsv.cvut.cz}).

Django bylo nainstalováno ve virtuálním prostředí, aby nebyl ovlivněn
zbytek GIS.labu. Při vývoji webové konzole byl využit odlehčený webový
server Djanga (vývojový server) a implicitní databáze SQLite. Pro
vytvoření základní struktury projektu a aplikace \textit{users} byly
použity vestavěné příkazy Djanga.

Balíčků pro autentizaci Djanga vůči \zk{LDAP} serveru existuje velké
množství. První ze zvažovaných knihoven byla
\textit{django-auth-ldap}, která provádí autentizaci uživatelů a při
jejich prvním přihlášení vytvoří účet v databázi Djanga. Nakonec byla
zvolena knihovna \textit{django\_python3\_ldap}, jež navíc obsahuje
možnost vytvořit všechny uživatelské účty zároveň spuštěním příkazu
\textsc{manage.py ldap\_sync\_users}. Kromě toho tuto knihovnu využívá
i platforma
Gisquick\footnote{https://github.com/gislab-npo/gislab/blob/master/system/roles/service-gisquick/files/static/django/settings\_custom.py},
která byla velkou inspirací v počátcích
vývoje. \textit{django\_python3\_ldap} navazuje na knihovnu
\textit{ldap3}.

Pro synchronizaci z Djanga do \zk{LDAP} byly provedeny pokusy využít
existující knihovny \textit{django-ldapdb} a
\textit{django-ldap-user-registration}. Ani jednu se nepodařilo
zprovoznit pro potřeby GIS.labu a proto byla vytvořena vlastní třída
\textsf{SyncDjangoLDAP}, která také staví na \textit{ldap3}.

\zk{LDAP} server v původní konfiguraci neumožňuje, aby samotní
uživatelé měnili údaje někoho jiného ani své vlastní. Proto byl
vytvořen modifikační soubor:

\begin{verbatim}
dn: olcDatabase={1}mdb,cn=config
changetype: modify
add: olcAccess
olcAccess: to * 
  by dn.exact="cn=admin,dc=gis,dc=lab" manage  
  by self write  
  by * read
\end{verbatim}

který do nastavení serveru přidal právo pro správce
\textit{cn=admin,dc=gis,dc=lab} upravovat všechny uživatele v
databázi, pro všechny uživatele upravovat své vlastní údaje a pro
ostatní právo čtení. Stejným způsobem bylo pozměněno již existující
nastavení práv k úpravě hesla, kde byly rozšířeny kompetence správce,
aby mohl upravovat všechny:

\begin{verbatim}
dn: olcDatabase={1}mdb,cn=config
changetype: modify
delete: olcAccess
olcAccess: {0}
-
add: olcAccess
olcAccess: {0}to attrs=userPassword
  by dn.exact="cn=admin,dc=gis,dc=lab" manage
  by self write
  by anonymous auth
  by * none
\end{verbatim}

Modifikační soubory byly využity při spuštění operace ldapmodify,
která slouží k úpravám konfigurace \zk{LDAP} serveru:
\begin{center}
\textsf{sudo ldapmodify -Q -Y EXTERNAL -f /cesta/k/modifikacnimu/souboru}
\end{center}

Změny byly aplikovány restartováním \textit{slapd}:
\begin{center}
\textsf{sudo service slapd restart}
\end{center}

V situaci, kdy jsou upravovány údaje v \zk{LDAP} skrze webovou
konzoli, je vytvořeno připojení s přihlašovacími údaji správce
\textit{cn=admin,dc=gis,dc=lab}, proto je dostačující nastavit
nejvyšší práva pouze pro něj.

Webové rozhraní má dvě hlavní části - uživatelskou a
administrátorskou. Administrátorská konzole vychází z vestavěné
správcovské konzole Djanga a je upravena potřebám
GIS.labu. Uživatelská konzole je vytvořená samostatně.

%zatim newpage, pak se to pořeší při dolaďování vzhledu
\newpage
\subsection{Struktura projektu}
Adresářová struktura projektu:
% tomuhle budu chtít dát hezčí design
% např. forest: https://tex.stackexchange.com/questions/5073/making-a-simple-directory-tree
% https://tex.stackexchange.com/questions/328886/making-a-directory-tree-of-folders-and-files/328890
\dirtree{%
.1 web\_admin\_console.			
	.2 project.
		.3 \_\_init\_\_.py.
		.3 ldap\_auth.py.
		.3 settings.py.
		.3 settings\_custom.py.
		.3 urls.py.
		.3 wsgi.py.
    .2 static.
    	.3 styles.css.
	.2 templates.
		.3 registration.
			.4 login.html.
		.3 base.html.
		.3 home.html.
		.3 password\_change.html.
		.3 signup.html.
		.3 user\_change.html.
	.2 users.
		.3 templatetags.
			.4 \_\_init\_\_.py.
			.4 auth\_extras.py.
		.3 \_\_init\_\_.py.
		.3 admin.py.
		.3 apps.py.
		.3 forms.py.
		.3 ldap\_sync.py.
		.3 models.py.
		.3 tests.py.
		.3 urls.py.
		.3 views.py.
	.2 db.sqlite3.
	.2 manage.py.
    }

\subsubsection{project}
Základní struktura projektu v Djangu byla vygenerována automaticky po
spuštění \textsf{django-admin startproject project} (viz
\ref{django-app}).

%settings_custom.py
Doplňující nastavení je definováno v novém souboru
\textbf{settings\_custom.py}, který je naimportován na konci původního
souboru \textit{settings.py}. V tomto souboru je v první řadě
definována \zk{URL} adresa, na které je webová konzole dostupná a
zaregistrovány používané aplikace - tedy vlastní aplikace
\textit{users} popsaná níže a knihovna \textit{django\_python3\_ldap},
která řídí komunikaci a synchronizaci s \zk{LDAP} serverem. Dále jsou
zde určeny cesty k nalezení šablon a statických souborů.

Rovněž je v tomto souboru definován vlastní uživatelský model a
autentizační backend využívající \zk{LDAP} server namísto
\textit{ModelBackend}, který je pro Django implicitní. Pro správnou
funkčnost knihovny \textit{django\_python3\_ldap} jsou nakonfigurovány
potřebné proměnné, mezi hlavními \zk{URL} \zk{LDAP} serveru, úroveň,
na níž se mají vyhledávat v \zk{LDAP} uživatelé či namapování
ekvivalentních atributů mezi \zk{LDAP} a Djangem. Kromě toho je na
konci volána vlastní funkce ze souboru \textit{ldap\_auth.py}, jež
provádí synchronizaci skupin do Djanga.

%ldap_auth.py
Soubor \textbf{ldap\_auth.py} sestává z jediné funkce
\textsf{custom\_sync\_user\_relations()}. Používaná knihovna
\textit{django\_python3\_ldap} při synchronizaci předpokládá existenci
atributu \textit{MemberOf} u uživatelského záznamu v databázi na
\zk{LDAP} serveru. Ten obsahuje informace o skupinách, jichž je
uživatel členem. GIS.lab server tento parametr u uživatelů nemá a
nastavení nebylo možné změnit, data o příslušnosti jsou ukládána pouze
na straně skupin. Proto byla napsána vlastní funkce, která
synchronizaci provádí.

Nejdříve je vytvořeno anonymní připojení k \zk{LDAP} serveru a získán
seznam existujících skupin. Ty jsou porovnány se skupinami v Djangu a
v případě nesrovnalostí je Django aktualizováno podle stavu na
\zk{LDAP} serveru. Stejným způsobem je provedena synchronizace
členství, kdy jsou postupně procházeny všechny skupiny a je testováno,
zda je uživatel jejich členem na \zk{LDAP} serveru. Po doběhnutí této
funkce jsou z pohledu skupin Django a \zk{LDAP} konzistentní.

%urls.py
Soubor \textbf{urls.py} v základní podobě obsahuje pouze \zk{URL}
adresu k vestavěné administrátorské konzoli Djanga. K ní jsou přidány
cesty ke správě uživatelů a k vlastní aplikaci \textit{users} a
definována šablona \textit{home.html} jako domovská stránka.

\subsubsection{users}

Tento adresář se skládá ze souborů tvořících Django aplikaci s názvem
\textit{users}, základní struktura byla vytvořena příkazem
\textsf{python manage.py startapp users}.

%models
V souboru \textbf{models.py} je definován vlastní uživatelský model
\textsf{CustomUser}, který dědí z třídy \textsf{AbstractUser} a
rozšiřuje ji o nový textový atribut \textsf{description}.

%forms
\textbf{forms.py} definuje třídy vlastních formulářů, které všechny
dědí z vhodných tříd Djanga a přidávají potřebnou
funkcionalitu. \textsf{CustomUserCreationForm} slouží k registraci
nových uživatelů, \textsf{CustomUserChangeForm} je použit při změně
osobních údajů. Oba formuláře mají nastaven model na
\textsf{CustomUser} a definován seznam polí, která se zobrazí při
registraci, resp. úpravě údajů. V třídě \textsf{FieldsRequiredMixin},
která je předána oběma jako parametr, jsou určeny povinné a nepovinné
atributy tak, aby odpovídaly atributům na \zk{LDAP} serveru. Povinné
jsou, vedle uživatelského jména a hesla, křestní jméno, příjmení a
emailová adresa. Nepovinné pak zůstává pouze pole popisu. U
registračního formuláře je navíc přetížena metoda \textit{save()} tak,
aby byl při vytváření nový uživatel uložen i na \zk{LDAP} server.

Posledním formulářem je \textsf{CustomAdminPasswordChangeForm}, jenž
slouží ke změně hesla. U něj je také přetížená funkce \textit{save()}
tak, aby nové heslo bylo upraveno i na serveru.

%admin
V souboru \textbf{admin.py} jsou veškeré změny, které upravují
standardní nastavení administrátorské konzole Djanga. Sestává ze dvou
tříd, jedné pro správu uživatelů a druhé pro správu skupin (rolí).

V \textsf{CustomUserAdmin} je připojen vlastní model
\textsf{CustomUser} a vlastní formuláře pro vytváření uživatele, změnu
údajů a hesla. Implicitní nastavení je upraveno. Zobrazují se jen
pole a filtry, které jsou pro administrátorskou konzoli potřebné. Při
manipulaci s uživatelským účtem přes administrátorskou konzoli se
nevolají metody formuláře, ale metody třídy \textsf{UserAdmin}. Proto
jsou zde přetíženy metody \textsf{save\_model()}, která vytváří či
upravuje uživatele, a \textsf{delete\_model()}, jež uživatele maže. Po
úpravě promítají tyto změny také na \zk{LDAP} server.

Podobně upravená je i třída \textsf{CustomGroupAdmin}. Ta sice
ponechává původní model \textsf{Group}, ale z atributů je zobrazen
jenom název. Metody \textsf{save\_model()} a \textsf{delete\_model()}
jsou přetížené stejně jako u administrace uživatelů a změny jsou
promítány na \zk{LDAP} server.

Na konci souboru jsou obě vlastní třídy zaregistrovány a popisy v
Django administraci jsou upraveny, aby lépe odpovídaly potřebám
GIS.lab konzole.

%ldap_sync
%TK:možná by to bylo přehlednější, kdybych tady k tomu připsala nějaký hodně stručný pseudokod? s tím, co se v které metodě synchronizuje?
Soubor \textbf{ldap\_sync.py} je zcela nově vytvořen a obsahuje
vlastní třídu \textsf{SyncDjangoLDAP}, jež provádí synchronizaci změn
z Djanga do \zk{LDAP}. Při její inicializaci je vytvořeno připojení k
\zk{LDAP} serveru s přihlašovacími údaji pro hlavního GIS.lab
administrátora se jménem \textit{cn=admin,dc=gis,dc=lab}, heslo je
uloženo v textovém souboru. V rámci destruktoru je připojení zrušeno.

Pro synchronizaci je využita externí knihovna ldap3, konkrétně operace
\textsf{ADD}, \textsf{MODIFY} a \textsf{DELETE}. Metoda
\textsf{change\_user()} zjistí, která data byla editována, a provede
tuto změnu i v \zk{LDAP}. Upravuje nejenom vlastní atributy uživatele,
ale i členství ve skupinách. Jen změna hesla má jinou metodu,
\textsf{change\_password()}, protože se jedná o atribut, který má
vlastní formulář pouze pro heslo.

Nového uživatele ukládá metoda \textsf{save\_user()}. Některé atributy
jsou zvoleny jako konstanty (např. identifikační číslo), protože se
jedná o interní hodnoty \zk{LDAP} a v Django neexistuje jejich
ekvivalent. Do budoucna by tento problém mělo vyřešit propojení s
Python knihovnou pro správu (viz \ref{python-knihovna}). Odstranění
účtu řídí \textsf{delete\_user()}, jež nejdřív odstraní všechny vztahy
uživatele a pak jej teprve smaže.

Skupiny lze pouze vytvořit přes \textsf{save\_group()} či odstranit v
metodě \textsf{delete\_group()}. Není umožněno provádět žádné úpravy,
protože jediným existujícím atributem je název role. Při tvorbě je
zvolena konstanta pro identifikační číslo skupiny, propojení s Python
knihovnou tuto situaci vyřeší (viz výše). Před odstraněním jsou
nejdříve zrušeny veškeré vztahy k uživatelům. V tomto ohledu se webová 
konzole chová odlišně od současné správy přes administrátorské příkazy. 
Příkaz \textsf{gislab-delgroup} nedovoluje smazat skupinu, pokud není 
prázdná.
 
%views
Soubor \textbf{views.py} obsahuje logiku propojující formuláře a
modely se zobrazením přes uživatelskou konzoli. Jedná se o tři pohledy
(třídy), jeden pro registraci (\textsf{SignUp}), druhý pro úpravu
uživatelských údajů (\textsf{ChangeUser}) a třetí pro změnu hesla
(\textsf{ChangePassword}). U všech je definováno, který formulář mají
použít, ve které šabloně je zobrazen a kam uživatele přesměrovat po
správném vyplnění a odeslání. U pohledu editace údajů je přetížená
funkce \textsf{form\_valid()} tak, aby navíc volala třídu
\textsf{SyncDjangoLDAP} a promítala změny i do \zk{LDAP}.

%urls
Posledním souborem v adresáři \textit{users} je \textbf{urls.py}, jenž
definuje \zk{URL} adresy ke třem výše zmiňovaným pohledům -
registraci, změně údajů a úpravě hesla.

\subsubsection{templates, templatetags, static}
Složka \textbf{templates} obsahuje šablony (templates), psané v jazyce
\zk{HTML}, které umožňují vypisovat vybraná data z modelů do
prohlížeče. Cesta k adresáři musí být registrována v souboru
\textit{settings.py} v proměnné \textsf{DIRS} u položky
\textsf{TEMPLATES}.

Pro přístup k proměnným a některým funkcím Pythonu slouží složené
závorky. V případě proměnných se jedná o závorky dvojité: \textsf{\{\{
  variable \}\}}.

%popis sablon
Django funguje na principu dědičnosti. Bázovou šablonou je
\textbf{base.html}. Vložením textu \textsf{\{\% extends "base.html"
  \%\}} na začátek jiné šablony, např. \textit{child.html}, je
nejdříve načtena šablona \textit{base.html}, definující základní
bloky, a až následně je k nim přidán obsah \textit{child.html}. Tímto
způsobem jsou omezeny duplicity v jednotlivých šablonách.

U tohoto projektu jsou například v bázové šabloně definovány navigační
prvky společné pro níže uvedené šablony, které z ní všechny dědí. Na
každé stránce se v horní části zobrazuje logo GIS.labu, které po
kliknutí přesměruje uživatele na hlavní stránku \textit{home.html}. V
případě přihlášeného uživatele se navíc zobrazuje tlačítko pro
odhlášení.

\textbf{signup.html} obsahuje jednoduchý interaktivní formulář s
informacemi, která pole jsou povinná pro platné vyplnění a jaké
podmínky musí splňovat. V situaci, kdy je registrace provedena
správně, je uživatel přesměrován na přihlašovací stránku, v opačném
případě je zobrazena chybová hláška a potřebná data je nutné opravit,
resp. doplnit.

Template \textbf{login.html} sestává z jednoduchého formuláře pro
vyplnění uživatelského jména a hesla. Při neúspěšném pokusu je vypsána
chyba, po zdárném vyplnění je uživatel přesměrován na domovskou
stránku \textit{home.html}.

První šablona, s níž se uživatel při zobrazení hlavní stránky setká,
je \textbf{home.html}. Ta ukazuje rozdílné výsledky nepřihlášenému
uživateli a přihlášenému. V prvním případě je dostupný rozcestník,
který jej navede na stránky přihlášení či k registraci. V druhém
případě, tedy pokud je již přihlášen, se zobrazí jeho osobní informace
a aktivní role. Přímo zde nemůže nic upravovat, ale pomocí tlačítka
\textsf{Edit} je přesměrován na stránku s úpravou osobních údajů.

Šablona \textbf{user\_change.html} obsahuje formulář pro editaci
osobních údajů, který umožňuje upravit jeden či více záznamů
naráz. Heslo se zde nezobrazuje, ale přes odkaz lze pokračovat na
stránku \textit{password\_change.html}, kde je možné provést jeho
změnu.

%static
Složka \textbf{static} umístěná v hlavním adresáři projektu obsahuje
soubor \textit{styles.css}, jenž popisuje způsob zobrazení elementů,
které jsou součástí jednotlivých šablon uživatelské konzole. To je
umožněno načtením tohoto souboru pomocí \textsf{\{\% load static \%\}}
do bázové šablony. K vytvoření designu byly použity kaskádové styly
(\zk{CSS}).

%template tags
Django umožňuje vytvořit si vlastní filtry. Ty jsou specifikovány v
souboru \textbf{auth\_extras.py} uloženém v adresáři aplikace
\textit{users/templatetags} a k šablonám jsou připojeny pomocí
\textsf{\{\% load auth\_extras \%\}}. Filtru \textsf{foo} lze předat
hodnotu proměnné \textit{var} a argument \textit{arg}. Poté, co je
%% ML: preteceni textu
filtr zaregistrován jako \textit{django.template.Library.filter()} a
definována žádaná funkce, je možné jej zavolat v šabloně příkazem:

\textsf{\{\{ var|foo:"arg" \}\}}

Pro potřeby zobrazení aktivních rolí uživatele v šabloně
\textit{home.html} byly vytvořeny dva filtry. První zjišťuje všechny
existující role v databázi Djanga a na něj navazuje druhý, který
určuje, zda je uživatel jejich členem. Výsledky jsou pak zobrazeny v
\zk{GUI}.

\subsubsection{db.sqlite3}
% https://www.tablesgenerator.com/latex_tables

Pro vývoj byla využita implicitní databáze Djanga SQLite. Z hlediska
uživatelů a~rolí jsou důležité především tři tabulky
\textit{auth\_group}, \textit{users\_customuser} a\newline
\textit{users\_customuser\_groups}.

Tabulka \textbf{auth\_group} obsahuje pouze názvy existujících rolí.

\begin{table}[H]
\centering
\begin{tabular}{@{}|c|c|c|@{}}
\toprule
\multicolumn{3}{|c|}{auth\_group} \\ \midrule
name & id & name \\ \midrule
type & integer & varchar(80) \\ \bottomrule
\end{tabular}
\caption{Atributy tabulky auth\_group}
\label{tab:auth-group}
\end{table}

Tabulka \textbf{users\_customuser} se skládá z osobních údajů uživatelů
včetně zašifrovaného hesla, data vytvoření, posledního přihlášení a
interních statusů Djanga.

\begin{table}[H]
\centering
\resizebox{\textwidth}{!}{%
\begin{tabular}{@{}|c|c|c|c|c|c|c|@{}}
\toprule
\multicolumn{7}{|c|}{users\_customuser} \\ \midrule
name & id & password & last\_login & is\_superuser & username & first\_name \\ \midrule
type & integer & varchar(128) & datetime & bool & varchar(150) & varchar(30) \\ \bottomrule
\end{tabular}%
}
\caption{Atributy tabulky users\_customuser 1/2}
\label{tab:users-customuser-1}
\end{table}

\begin{table}[H]
\centering
\resizebox{\textwidth}{!}{%
\begin{tabular}{@{}|c|c|c|c|c|c|c|@{}}
\toprule
\multicolumn{7}{|c|}{users\_customuser} \\ \midrule
name & last\_name & email & is\_staff & is\_active & date\_joined & description \\ \midrule
type & varchar(150) & varchar(254) & bool & bool & datetime & text \\ \bottomrule
\end{tabular}%
}
\caption{Atributy tabulky users\_customuser 2/2}
\label{tab:users-customuser-2}
\end{table}

Tabulka \textbf{users\_customuser\_groups} propojuje obě výše zmíněné
tabulky, tj. příslušnost uživatelů k jednotlivým skupinám.

\begin{table}[H]
\centering
\begin{tabular}{@{}|c|c|c|c|@{}}
\toprule
\multicolumn{4}{|c|}{users\_customuser\_groups} \\ \midrule
name & id & customuser\_id & group\_id \\ \midrule
type & integer & integer & integer \\ \bottomrule
\end{tabular}
\caption{Atributy tabulky users\_customuser\_groups}
\label{tab:users-customuser-groups}
\end{table}

\subsection{Spuštění projektu}
Ve finální verzi bude webová konzole zaintegrována do GIS.labu pomocí
Docker kontejneru jako samostatná služba. Při odevzdání je projekt
ještě ve vývojové fázi. Pro jeho spuštění je třeba mít nainstalovanou
platformu GIS.lab alespoň ve virtuální režimu. Projekt si uživatel
může sestavit sám či může využít připravený skript, který je dostupný
v
repozitáři. \footnote{https://github.com/ctu-geoforall-lab-projects/dp-kulovana-2019}

Shellový skript \textbf{setup\_script.sh} připraví projekt ke
spuštění, jediné co musí uživatel udělat, je upravit proměnnou
\textsf{PROJ\_PATH}, která obsahuje cestu k adresáři, do něhož má být
projekt uložen. Poté jen stačí spustit skript příkazem:
\begin{center}
\textsf{./setup\_script.py}
\end{center}
Nejdříve je vytvořen adresář projektu, poté je nainstalován Python 3 s
programem pip a virtuálním prostředím. V dalším kroku je spuštěno
virtuální prostředí a v něm je nainstalováno Django a knihovna
\textit{django\_python3\_ldap}. Pak je vytvořena základní struktura
projektu a aplikace. Repozitář diplomové práce je naklonován do
dočasného adresáře, odkud jsou všechny upravené či nově vytvořené
soubory překopírovány na správnou lokaci v projektu. Tento dočasný
adresář je následně smazán. Na konec je provedena migrace databáze.

Tím je projekt téměř připraven. Webové rozhraní bude dostupné na
adrese, která je uvedena v nastavení \textit{settings\_custom.py} v
proměnné \textsf{ALLOWED\_HOSTS}. Proto musí uživatel tuto cestu
upravit tak, aby odpovídala adrese serveru. Pro spuštění projektu je
třeba přejít do složky projektu a aktivovat virtuální prostředí:
\begin{center}
\textsf{source virenv/bin/activate}
\end{center}

Poté již lze spustit vývojový server:

\begin{center}
\textsf{python manage.py runserver 0:8080}
\end{center}

Při zobrazení stránky uvedené v proměnné \textsf{ALLOWED\_HOSTS} a
portu 8080 (např. http://b802-01.fsv.cvut.cz:8080) se uživatel dostane
na domovskou stránku webové konzole.

\subsection{Budoucí vývoj}
\label{python-knihovna}

%ověřování přes maily
V době odevzdání probíhá proces registrace tak, že po odeslání
validního formuláře je okamžitě vytvořen uživatelský účet v Djangu i
na \zk{LDAP} serveru.

\begin{figure}[H] \centering
    \includegraphics[width=80pt]{./pictures/my_console_current_version_cz.png}
    \caption[Vytváření uživatele - současný stav]{Vytváření uživatele - současný stav (zdroj:
	\href{}{Tereza Kulovaná})}
    \label{fig:admin-current}
\end{figure}
  
V konečné formě by mělo mezi jednotlivými činnostmi probíhat potvrzení
přes email. Konkrétně přímo po registraci bude uživateli odeslán email
na vyplněnou adresu, který bude nutné před následujícími akcemi
potvrdit. Poté bude vytvořen účet v Djangu s neaktivním statusem,
který uživateli zabraňuje se do systému přihlásit. Správce obdrží
email s žádostí o vytvoření uživatelského účtu, jenž bude přímo
obsahovat volby pro potvrzení a zamítnutí. Pokud bude požadavek
zamítnut, administrátor vyplní zdůvodnění tohoto rozhodnutí, účet bude
z Djanga smazán a uživatel obdrží vysvětlující zprávu. V případě
vyhovění žádosti se účet stane aktivním a vytvoří se jeho ekvivalent v
\zk{LDAP}. Uživatel bude o kladném rozhodnutí spraven emailem.
  
\begin{figure}[H] \centering
    \includegraphics[width=250pt]{./pictures/my_console_final_version_cz.png}
    \caption[Vytváření uživatele - finální stav]{Vytváření uživatele - finální stav (zdroj:
	\href{}{Tereza Kulovaná})}
    \label{fig:admin-final}
\end{figure}

Ověřování platnosti emailové adresy bude doplněno i do části
uživatelské konzole, jež umožňuje editaci osobních údajů.

%python knihovna
Vývoj knihovny pro správu uživatelů psané v jazyce Python má počátek v
roce 2015. Její vznik byl iniciován především proto, aby nahradila
stávající shellové skripty, protože vývojářům GIS.labu je bližší
Python a následné úpravy pro ně budou tímto způsobem snadnější. Neméně
důležitá je i možnost propojení s webovým rozhraním, a tak byly práce
na této knihovně nedávno po delší odmlce obnoveny.

Aktuálně jsou zpracovány skripty pro tvorbu, úpravu a mazání
uživatelských účtů a správu známých zařízení v síti.  Před propojením
s webovou konzolí je bude potřeba ještě dokončit a vytvořit nové pro
správu skupin (rolí).
  
% loggování a zprávy
Přes opakované pokusy se nepodařilo zprovoznit vypisování informačních
zpráv (tzv. logů) pro úroveň DEBUG, které by usnadnily vývoj. Aktuálně
jsou dostupné pouze hlášky pro úroveň INFO a vyšší. Před dalším
postupem bude třeba tuto problematiku hlouběji prozkoumat a vyřešit.

V rámci vlastních funkcí jsou vypisovány informační logy v podobě
prostého textu. Ty dostanou vhodnější formátování a budou doplněny o
datum a čas, kdy daná činnost proběhla. Pro lepší informovanost o dění
budou přidány chybové hlášky. Veškeré informace, které se aktuálně
ukazují v konzoli, budou ukládány do souborů.

Administrátorská konzole, která vychází z konzole Djanga, obsahuje
informační zprávy, jež se správci zobrazují ve webovém
prohlížeči. Tato funkcionalita bude doplněna i pro uživatelskou
konzoli.

% ldap_auth.py
Synchronizace skupin do Djanga je realizována funkcí
\textsf{custom\_sync\_user\_relations()}. V rámci úprav kódu se stane
členskou metodou nové třídy.

% žádost o role
O role, které budou uživatele opravňovat k využití jednotlivých
balíčků GIS.labu, si bude moci uživatel zažádat sám. Nyní se v
uživatelské konzoli klient pouze dozví, které role jsou pro něj
aktivní. Ve finální verzi si bude moci uživatel vybrat zvolené role
během registrace či si o jejich změnu zažádat přes uživatelskou
konzoli. Vyhovění či zamítnutí požadavku provede administrátor a
uživatel bude informován emailem. Podobně bude implementována i žádost
o navýšení kapacity databáze a další služby popsané v kapitole GIS.lab
(\ref{vision}).

%docker
%% ML: dopsat ...
Docker ještě doplním.
